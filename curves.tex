% arara: xelatex: {shell: yes}
% %arara: biber
% %arara: xelatex: {shell: yes}
% %arara: xelatex: {shell: yes}

\documentclass[12pt]{article}

\usepackage{hyperref} % гиперссылки

\usepackage{tikz} % картинки в tikz
\usetikzlibrary{arrows.meta} % tikz-прибамбас для рисовки стрелочек подлиннее

\usepackage{microtype} % свешивание пунктуации

\usepackage{array} % для столбцов фиксированной ширины

\usepackage{indentfirst} % отступ в первом параграфе

\usepackage{sectsty} % для центрирования названий частей
\allsectionsfont{\centering}

\usepackage{amsmath} % куча стандартных математических плюшек
\usepackage{amssymb} % символы
\usepackage{amsthm} % теоремки

\usepackage{comment} % добавление длинных комментариев

\usepackage[top=2cm, left=1.2cm, right=1.2cm, bottom=2cm]{geometry} % размер текста на странице

\usepackage{lastpage} % чтобы узнать номер последней страницы

\usepackage{enumitem} % дополнительные плюшки для списков
%  например \begin{enumerate}[resume] позволяет продолжить нумерацию в новом списке

\usepackage{caption} % что-то делает с подписями рисунков :)

\usepackage{qcircuit} % для рисовки квантовых диаграмм
\usepackage{physics} % бракеты

\usepackage{answers} % разделение условий и ответов в упражнениях


\usepackage{fancyhdr} % весёлые колонтитулы
\pagestyle{fancy}
\lhead{Куда кривая вывезет}
\chead{}
\rhead{КЛШ-2022}
\lfoot{}
\cfoot{}
\rfoot{\thepage/\pageref{LastPage}}
\renewcommand{\headrulewidth}{0.4pt}
\renewcommand{\footrulewidth}{0.4pt}



\usepackage{todonotes} % для вставки в документ заметок о том, что осталось сделать
% \todo{Здесь надо коэффициенты исправить}
% \missingfigure{Здесь будет Последний день Помпеи}
% \listoftodos — печатает все поставленные \todo'шки



\usepackage{booktabs} % красивые таблицы
% заповеди из докупентации:
% 1. Не используйте вертикальные линни
% 2. Не используйте двойные линии
% 3. Единицы измерения - в шапку таблицы
% 4. Не сокращайте .1 вместо 0.1
% 5. Повторяющееся значение повторяйте, а не говорите "то же"



\usepackage{fontspec} % что-то про шрифты?
\usepackage{polyglossia} % русификация xelatex

\setmainlanguage{russian}
\setotherlanguages{english}

% download "Linux Libertine" fonts:
% http://www.linuxlibertine.org/index.php?id=91&L=1
\setmainfont{Linux Libertine O} % or Helvetica, Arial, Cambria
% why do we need \newfontfamily:
% http://tex.stackexchange.com/questions/91507/
\newfontfamily{\cyrillicfonttt}{Linux Libertine O}

\AddEnumerateCounter{\asbuk}{\russian@alph}{щ} % для списков с русскими буквами
\setlist[enumerate, 2]{label=\asbuk*),ref=\asbuk*}

%% эконометрические сокращения
\DeclareMathOperator{\Cov}{Cov}
\DeclareMathOperator{\Arg}{Arg}
\DeclareMathOperator{\Corr}{Corr}
\DeclareMathOperator{\Var}{Var}
\DeclareMathOperator{\E}{\mathbb{E}}
\newcommand \hVar{\widehat{\Var}}
\newcommand \hCorr{\widehat{\Corr}}
\newcommand \hCov{\widehat{\Cov}}
\newcommand \cN{\mathcal{N}}
\let\P\relax
\DeclareMathOperator{\P}{\mathbb{P}}

\usepackage{multicol}

\usepackage[bibencoding = auto,
backend = biber,
sorting = none,
style=alphabetic]{biblatex}

\addbibresource{forecast_everything.bib}



% делаем короче интервал в списках
\setlength{\itemsep}{0pt}
\setlength{\parskip}{0pt}
\setlength{\parsep}{0pt}




\Newassociation{sol}{solution}{solution_file}
% sol --- имя окружения внутри задач
% solution --- имя окружения внутри solution_file
% solution_file --- имя файла в который будет идти запись решений
% можно изменить далее по ходу
\Opensolutionfile{solution_file}[all_solutions]
% в квадратных скобках фактическое имя файла

% магия для автоматических гиперссылок задача-решение
\newlist{myenum}{enumerate}{3}
% \newcounter{problem}[chapter] % нумерация задач внутри глав
\newcounter{problem}[section]

\newenvironment{problem}%
{%
\refstepcounter{problem}%
%  hyperlink to solution
     \hypertarget{problem:{\thesection.\theproblem}}{} % нумерация внутри глав
     % \hypertarget{problem:{\theproblem}}{}
     \Writetofile{solution_file}{\protect\hypertarget{soln:\thesection.\theproblem}{}}
     %\Writetofile{solution_file}{\protect\hypertarget{soln:\theproblem}{}}
     \begin{myenum}[label=\bfseries\protect\hyperlink{soln:\thesection.\theproblem}{\thesection.\theproblem},ref=\thesection.\theproblem]
     % \begin{myenum}[label=\bfseries\protect\hyperlink{soln:\theproblem}{\theproblem},ref=\theproblem]
     \item%
    }%
    {%
    \end{myenum}}
% для гиперссылок обратно надо переопределять окружение
% это происходит непосредственно перед подключением файла с решениями



\theoremstyle{definition}
\newtheorem{definition}{Определение}



\begin{document}

\tableofcontents{}

\section*{Анонс}
...

\newpage
\setcounter{section}{0}
\section{Парабола}

Три алгебраических вида. Важно уметь быстро строить из любого вида!

\[
y = ax^2 + bx + c  
\]
\[
y = a(x - x_{\text{в}}) + y_{\text{в}}
\]
\[
y = a(x - x_1) (x-x_2)  
\]

Совет: рисуйте сразу, не переводя из одного вида в другой. 

Вопрос: правда ли, что все круги одинаковой формы, но разного размера?

Вопрос: правда ли, что все параболы одинаковой формы, но разного размера?

Подумайте о $y=x^2$ и $y=6x^2$.

Два геометрических определения.

Парабола — множество точек, находящихся на одинаковом расстоянии от заданной точки $F$ 
и заданной прямой $d$. Точка $F$ называется фокусом, а прямая $d$ — директрисой. 

Упражнение. Даны фокус $F$ и директриса $d$. 
Как геометрически построить какую-нибудь точку на параболе?

Парабола — кривая, отражающая параллельно идущие лучи в одну точку $F$. 

Доказательство того, определение через множество точек обладает свойством фокусировки лучей. 

Шок-контент. Все параболы одинаковой формы! 
Ведь при увеличении можно произвольным образом менять расстояние между фокусом и директрисой, а именно им всё и определяется. 
Алгебраически, $y=6x^2$, $6y = 6^2 x^2$, $\tilde y = \tilde x^2$.

Упражнение. Дан фокус $F$ и директриса $d$. Как наиболее просто выбрать оси?
Запишите уравнение параболы в выбранных осях. 

Упражнение. Дана парабола $y=x^2$. Найдите фокус и директрису. 


Упражнение. Дана парабола $y=2x^2 + 6x + 7$. Найдите фокус и директрису. 

О школьниках: на первом занятии было 17 человек. 

\section{Заметай}

Вспоминаем, что парабола сама построится в виде огибающей, если нарисовать все касательные. 

Вопрос: как можно описать прямую?

Вопрос: как можно описать дружное семейство прямых?

Прямая определяется двумя точками. Если добавить параметр $a$ в координаты этих двух точек, то получится семейство прямых!

Как убить время и заработать деньги с помощью параболы?

Упражнение. Нарисуйте семейство прямых, проходящих через $L(0, a)$ — $R(10 - a, 0)$. 
Запишите формулой это семейство. Найдите (п)огибающую визуально и аналитически.
Находить уравнение огибающей проще в новых координатах, $x' = y - x$, $y'=x+y$.

google: envelope / string art / огибающая / изонить

Рисуем прямые или отрезки в любом количестве. Размечаем все прямые с равным шагом на каждой прямой. 
Соединяем размеченные точки на паре прямых семейством прямых, получаем огибающую семейства. Повторяем с разными парами прямых, получаем разные огибающие.

Упражнение. Нарисуйте семейство прямых, проходящих через $L(a, a)$ — $R(10 - a, 0)$. 
Запишите формулой это семейство. Найдите (п)огибающую визуально и аналитически.
Подумайте, в каких ортогональных координатах удобнее находить уравнение огибающей. 

Снова шок-контент: форма огибающей семейства не зависит от того, 
взяты ли ортогональные оси или прямые под углом в один градус для построения семейства огибающих.

doodle: параболы между лучами пучка прямых, параболы в шестигольнике. 

Можно делать поделки или NFT :)

О школьниках: на первом занятии было 17 человек. 

\section{Разрезай, Коси и Заметай} 

аддитивность, принцип Кавальери, принцип Мамикона

Площадь окружности с помощью разрезов. 

Коси

Скошенная колода карт.

Заметай.

Обходим эллипс касательным отрезком, даны длины полуосей и длина отрезка. Находим площадь.

Трактриса. 

Площадь кольца — два способа. Можно вычесть окружности, можно обойти касательным отрезком меньшую окружность. 


\section{Загоночная работа}




\newpage

\section{Лог. КЛШ-2022}

\begin{enumerate}
  \item 
\end{enumerate}

В теховском файле \verb|\newpage| стоит, чтобы легко было скопировать секцию, для печати двух копий подряд на одном листе.
Это позволяет экономить бумагу и время при печати :)

\subsection{Плакат}





\Closesolutionfile{solution_file}

% для гиперссылок на условия
% http://tex.stackexchange.com/questions/45415
\renewenvironment{solution}[1]{%
         % add some glue
         \vskip .5cm plus 2cm minus 0.1cm%
         {\bfseries \hyperlink{problem:#1}{#1.}}%
}%
{%
}%



\section{Решения}
\input{all_solutions}


\section{Источники мудрости}

\todo[inline]{передалать потом в bib-файл}

\begin{enumerate}
\item \url{https://math.stackexchange.com/questions/475666/}
\item \url{https://en.wikipedia.org/wiki/Parabola}
\item \url{http://www.physicsinsights.org/}
\item \url{https://en.wikipedia.org/wiki/Hyperbola}
\item \url{https://www.mathed.page/parabolas/geometry/}
\item \url{https://en.wikipedia.org/wiki/Ellipse}
\end{enumerate}

\printbibliography[heading=none]


\end{document}
